\documentclass[10pt, a4paper]{article}
\usepackage[utf8]{inputenc}

\title{Game Tree Searching by Min / Max Approximation In Review}
\author{Nathan Findley}
\date{March 2017}

\begin{document}

\maketitle

\begin{abstract}

A new game\cite{lamport94} tree search technique called Min / Max Approximation was introduced
by Ronald L. Rivest in 1986.  An iterative approach, it is used to determine
which leaf should be futher expanded so that the root will have a value that
approaches the true values of a fully expanded tree.  When evaluated in 
comparison with minmax alpha-beta pruning
using iterative deepening over the course of nearly 1,000 trials, Min / Max
approximation was seen to be either a strong contender when limiting turns by move count 
or a slightly weaker contender when limiting turns by time.

\end{abstract}

\section{Problem} 

When designing an algorithm to handle all possible branches of all possible
plays within any given game, the immediate limiting factors are time and computation: a perfect
view of the entire tree of gameplay for a particular game is not feasible.

\section{Goal} 

Min / Max Approximation is attempting to solve this problem.  Unlike minimax
alpha-beta pruning with iterative deepening which moves down the tree with a uniform depth, 
Min / Max approximation follows a
given branch in a tree because it appears to be beneficial.  As a result, different
levels of depth are reached in different branches of the partial tree.  Essentially one branch is explored
until the root node's value is considered better when exploring a different tip.
At that point in time, search switches to the seemingly more valuable tip.  
Time permitting, the depth of the most valuable branch will reach a terminating leaf.  As to which tip should 
be explored, Min / Max approximation is presented as a possible answer.

Penalties are introduced in order to determine which tip to pursue.  "Bad" tips of the tree
are weighted with a nonnegative penalty that distinguishes them from "good" tips.
The overall penalty for a given tip is the sum of all penalties tracing up to the original root of the partial tree.
This calculation of a viable tip is what distinguishes this algorithm from minimax alpha-beta: the tip that causes
the root the have the lowest penalty is the one that is further explored. As soon as a tip's
weight drops its penalty score below any other tip in the partially expanded
tree, that more favorable tip is then explored. According to the author, one of Min / Max approximations 
strengths is to encourage branches that have secondary lines of play that might also be
winnable since the penalty calculation is determined cumulatively rather than
just using a child with the highest value.

\section{Results} 

Connect four was used to explore whether or not this is a viable algorithm in comparison
with minimax alpha-beta pruning.  After nearly 1000 trials, the author felt confident
that this method could be further explored and may possibly be incorporated with existing
techinques at the time. Overall Min / Max Approximation appears more effective than traditional 
minimax alpha-beta pruning with iterative deepening as long as the number of moves 
is the limiting resource when comparing the two methods.  Unfortunately game rounds are typically
constrained by time, preventing this method from immediately being the best option available.

\begin{thebibliography}{9}

\bibitem{lamport94}
  Leslie Lamport,
  \emph{\LaTeX: a document preparation system},
  Addison Wesley, Massachusetts,
  2nd edition,
  1994.

\end{thebibliography}

\end{document}
