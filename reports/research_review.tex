\documentclass[10pt, a4paper]{article}
\usepackage[utf8]{inputenc}

\title{Developments in AI Planning and Search}
\author{Nathan Findley}
\date{June 2017}

\begin{document}

\maketitle

\begin{abstract}

The subset of AI research focused on planning has been actively pursued for the past three decades.  
Historically there were various periods, such as the AI winter of 1987 - 1993, during which research
stagnated due to lack of governmental funding for research.  Nonetheless improvements within the field
of AI continued sporatically and have blossomed within the last few years.  As recently as 2015 Jack Clark
of Bloomberg indicated\cite{jack} that Google's internal usage of AI had exploded to more than 2700 projects.

\end{abstract}

Planning is the subset of AI concerned with taking both a set of initial states and a set of end states and finding
a reasonable way of determining how to concretely tranform one of these sets to the other.  This particular mindset
was established with the seminal work of Richard E. Fikes and Nils J. Nilsson in 1971 when they published their methodology
known as STRIPS.  This approach resulted in a simplification of the established approach:  when dealing with planning methods 
the states that change - the Effect Axioms - as well as the unchanged states - the Frame axioms - as a result of a particular action were, 
up until the introduction of STRIPS, both required in order to solve the planning problem\cite{derek}.  The problem of tracking the Frame axioms as well
as determining that the Frame axioms are in fact exhaustive, turns out to be difficult. 
"In STRIPS, we surmount these difficulties by separating entirely the processes of theorem proving from those of searching through a space of world models."\cite{richard}
Unfortunately as time went on, solving problems that were little more than simple puzzles didn't lead to approaches that could actually
solve real world problems.

In 1995 Graphplan takes a planning problem expressed in STRIPS and searches for a goal state.  Relative to other techiniques at the time, its
authors argued that "Graphplan always returns a shortest-possible partial-order plan, or states that no valid plan exists".\cite{ablum}

Heuristics Search Planning was later established as a good means of planning by whereby "informative heuristic function could be constructed automat-
ically"\cite{derek}.  Determining planning routes is then left to a relaxed plan approach.

In 1998 the PDDL was created in order to make the International Planning Completition possible. 
This common language "fosters far greater reuse of research and allows more direct comparison of systems and approaches"\cite{mfox}
which accelerates research and eases the sharing of knowledge.  Planning problems typically use PDDL.

In the early 1990s a fissure developed in the AI community over whether general purpose algorithms, like those found in classical planning, or domain
specific algorithms, which typically produced better results, but were not widely applicable, were better.  As classical planning methods are relaxed
solutions to real world problems become more attainable.  Further research into methods that incorporate new data during the execution of the planning algorithm as well
as algorithms that deal with external uncertainties are likely to be important parts of the future of AI planning research.

\begin{thebibliography}{9}

\bibitem{ablum}A. Blum and M. Furst \emph{"Fast planning through planning graph analysis"} Artificial intelligence. 90:281–300, 1997 
\bibitem{derek}Derek Long, Maria Fox, \emph{"Progress	in AI Planning Research and Applications"}, UPGRADE Vol. III, No. 5, October 2002
\bibitem{jack}Jack Clark, Bloomberg News, \emph{"Why 2015 Was a Breakthrough Year in Artificial Intelligence"}, 8 December 2015.
\bibitem{mfox}M. Fox, D. Long, \emph{"PDDL+: Modeling continuous time dependent effects"}, Proceedings of the 3rd International NASA Workshop on Planning and Scheduling for Space.
\bibitem{richard}Richard E. Fikes, Nils J. Nilsson, \emph{"STRIPS: A New Approach to the Application of Theorem Proving to Problem Solving"}, 1 September 1971, p 190

\end{thebibliography}

\end{document}
